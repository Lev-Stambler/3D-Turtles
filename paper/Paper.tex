\documentclass[11pt,titlepage]{article}
\usepackage{TCSToolkit}

\newcommand{\dofloor}[1]{\lfloor#1\rfloor}
\newcommand{\doceil}[1]{\lceil#1\rceil}

%%%%% Stuff you can change %%%%%%%%%%%%%%%%%%%%%%%%%%%%%%%%%%
\newcommand{\myname}{Lev Stambler}
%

%% Custom Commands
\newcommand{\rationalPeriod}{{T^j}'}
\newcommand{\rationalPeriodTotal}{{T}'}
\newcommand{\statePeriod}{T^j}
\newcommand{\rationalAngleJ}[2]{{\theta^{#2}_{#1}}}
\newcommand{\rationalAngle}[1]{{\theta^j_{#1}}}
\newcommand{\rationalAngleNotime}{{\theta}^j}
\newcommand{\state}[2]{{s_{#1}}^{#2}}
\newcommand{\stateNotime}[1]{s^{#1}}
\newcommand{\totalPeriod}{T}
\newcommand{\totalState}{\pmb{s}_i}
\newcommand{\totalStateNotime}{\pmb{s}}
\newcommand{\comb}{Comb}
\newcommand{\combWithState}[1]{Comb\left(s^1_{#1}, s^2_{#1}, ..., s^k_{#1}\right)}
\newcommand{\inclIndic}[2]{{\mathrm{incl}_{#1}^{#2}}}
\newcommand{\numbToAngle}{\frac{2\pi}{b^j}}
\newcommand{\numbToAngleNoJ}[1]{\frac{2\pi}{b^{#1}}}
\newcommand{\combSingleTermCos}[1]{\prod_{j=1}^k \cos\left(\numbToAngle \cdot \theta^j\right) ^ {\inclIndic{#1}{j}}}
\newcommand{\digSumPeriodic}{\sigma^j}
\newcommand{\digSumPeriodicNoJ}{{\sigma}}
\newcommand{\totalOverRationalFrac}{\frac{\totalPeriod}{\rationalPeriodTotal}}
\newcommand{\changeTotalOneDim}{\Delta P_{0, T}^d}
\newcommand{\digit}{\mathrm{digit}}
\newcommand{\commonBase}{{\pmb{b}}}
\newcommand{\numbToCommon}{\frac{\commonBase}{b^j}}
\newcommand{\commonToAngle}{\frac{2\pi}{\commonBase}}
\newcommand{\sinOrCos}{\phi^j}
\newcommand{\combSingleTerm}[1]{\prod_{j=1}^k \sinOrCos\left(\numbToAngle \cdot \theta^j\right) ^ {\inclIndic{#1}{j}}}


%
% Final tip: you can reference HW5 in your TeX using \Cref{hw:5}; 
% and, you can reference HW5.Problem3 in your TeX using \Cref{prob:5.3}
%


%%%%% Section-renaming code by egreg
\makeatletter
% we use \prefix@<level> only if it is defined
\renewcommand{\@seccntformat}[1]{%
  \ifcsname prefix@#1\endcsname
    \csname prefix@#1\endcsname
  \else
    \csname the#1\endcsname\quad
  \fi}
% Now we define our homework section prefixes
% \newcommand\prefix@section{Homework \thesection: }
% \newcommand{\prefix@subsection}{Problem \thesubsection: }
% \newcommand{\prefix@subsubsection}{Section \thesubsubsection: }
\makeatother
%%%%%




\begin{document}

\title{Drawing Math}

\author{\myname}

\date{\today}

\maketitle

\pagebreak
\section{Introduction}
- Numberphile video
- I was bored and this looked fun
- I wasn't paying attention in one of my lectures...
- This paper is purely for funs... for now

\section{Background}
- Digital math
- Euler's formula
- I payed attention in some of my lectures

\section{Definitions and questions}
\subsection{Definitions}
Say you (yes you!) had a turtle living in $D$ dimensional Euclidean
space and in discrete time. At time step $i$, where $i \in \Z$ and $i > 0$,
the turtle has position $p_i \in \R^D$.
Then, lets define $\Delta p_{i+1} = p_{i+1} - p_i$; in other words, $\Delta p_{i+1}$ is the change in position from time $i$ to $i + 1$.

Now say that the turtle's movement is determined by $k$ seed parameter drawn from
the same set. Then, for some state space $\mathcal{S}$,
define $s_i^j \in \mathcal{S}$ to be some arbitrary
state associated with timestamp $i$ for the $j$th seed parameter where $j \in [k]$.
Also, define $\pmb{s}_i = (s_i^1, s_i^2, ..., s_i^k)$.
Next we will define a set of functions $SU^j: \mathcal{S} \rightarrow \mathcal{S}$
(for $S$tate $U$pdater) such that $s_{i + 1}^j = SU^j(s_i^j, i)$. Note that for
$j, a \in [m]$ where $j \neq a$, $s_{i+1}^j$ is determined solely by $s_i^j$ and $i$ and not
$s_i^a$.

Now that we have our machinery built up, lets define $Comb: \mathcal{S}^k \rightarrow \R^d$ such that
$$
  \Delta p_{i + 1} = \combWithState{i + 1}.
$$
In other words, $Comb$ takes in the state of each seed and
returns an update to the position of the turtle.

Finally, let us define
$$
  \Delta P_{a, b} = \sum_{i = a}^{b} \Delta p_i.
$$
In other words, $\Delta P_{a, b}$ is the change in position from timestep $a$ to $b$.

\subsection{The problem}
Say we are given, $Comb$, hh$SU^j$, $p_0$, and $s^j_0$ for all $j \in [k]$.
Informally, the question is whether the turtle draws a ``closed" shape or not.\\
More formally, is there some period $T$ such that 
$$
  p_{i + \ell T} = p_{i}
$$
for $i, \ell \in \N$.
Then, note that if there exists a period $T$ such that 
$\Delta P_{i, i + \ell T} = 0$ for all $i, \ell \in N$, 
$p_{i + \ell T} = p_{i}$ and the turtle forms a closed shape.

% TODO: ask closure question
% TODO: show result that if $(s_{a}^1, ... ) = (s_{b}^1,...)$ then $b-a$ is a period for the state
% and Delta p_a = p_b and is thus invariant under mod b-a. So, if Delta p_{b-a} \neq 0 
% we have that the turtle will not move in a closed shape. If Delta p_{b-a} = 0, it will  

\subsection{Specifying the task ahead of us}
For our case, we consider $Comb, SU_i^j$ to all be memoryless (i.e.\ there output
is uniquely determined by the current input). So, we can simplify the overall question.
If, $\pmb{s}_i = \pmb{s}_{i + \ell T}$ for some $T \in \N$ and all $i \in \N$, then 
$\Delta p_{i} = \Delta p_{i + \ell T}$. So then,
$\Delta P_{i, i + \ell T} = \Delta P_{i, i + \ell' T}$ for all $\ell, \ell' \in \N$.
Thus, $T$ is a period of the change in position. We can thus
break down our problem into two parts:
\begin{enumerate}
  \item Finding the period, $T$, of the state $\pmb{s}$.
  \item Checking whether $\Delta P_{i, i + T} = 0$.
\end{enumerate}

\subsection{Some more restrictions on our problem}
We further restrict the problem by only considering $\mathcal{S} = \N^4$ where for $(n, d, b, \theta) \in \mathcal{S}$,
$n$ is the numerator of a rational in fraction form, $d$ is the denominator, 
$b$ is the base (i.e.\ base 10, base 12, etc.),
and $\frac{2\theta}{b \pi}$ is an ``angle" associated with the state.

Then, let $\sinOrCos: \R \rightarrow \R$ equal $\cos$ or $\sin$. 

Now, we will only consider
$$
  SU_i^j(n, b, d, \theta) = (n, b, d, \theta + \mathrm{digit}(n, b, d, i) \mod b).
$$
where $\mathrm{digit}(n, b, d, i)$ gives us the $i$th digit of the decimal expansion of $\frac{n}{d}$
in base $b$. For the sake of convenience, we will use the word ``rational parameter" 
instead of ``seed parameter" from here on out.

Moreover, we consider the case where
$$
  Comb((., ., ., ., \theta^1), (., ., ., ., \theta^2), ..., (., ., ., ., \theta^k)) =
    \left(\combSingleTerm{1}, ..., \combSingleTerm{D}\right)
$$
where $\mathrm{incl}_d^j \in \set{0, 1}$ for $d \in [D]$ indicates whether to include a given
$x \in R$ determined by rational parameter $j$ for position update in the $d$th dimension.

Finally, for simplicity's sake, assume that $\theta = 0$ for all $(n, b, d, \theta) \in \pmb{s}_0$,
$n < d$, and $\frac{n}{d}$'s decimal expansion is periodic after some $N \geq 0$ decimal places
and does not terminate in base $b$.

Also, lets set 
$$\commonBase = \lcm{(n, b, d, \theta) \in \pmb{s}_0} b.$$
In other words,
$\commonBase$ can be thought of as a ``common base" among all rational parameters.

\subsubsection{Some intuition}
While the restrictions may seem arbitrary, they aptly match our original problem statement.
The original problem statement derives a spherical change in position based off of a rational number's
digit at a particular timestep. The polar change in position also has a fixed radius. Translating
from a polar to cartesian update then only requires products of $\sin$s and $\cos$s. See \cite{NDimSphericalCoord}
for more details.

Take the three dimensional case for instance. The turtle's update cartesian space
is given by 
\begin{align*}
  x &= \cos(\alpha)\\
  y &= \sin(\alpha) \cos(\beta)\\
  z &= \sin(\alpha) \sin(\beta)
\end{align*}
where $\alpha = \numbToAngleNoJ{1} \cdot \theta^1$ and 
$\beta = \numbToAngleNoJ{2} \cdot \theta^2$. We can thus see that our definition
of $\comb$ captures the three dimensional case.

% TODO: clear up.

% TODO: put your own diagram in!!


\section{Does it close?}
In understanding whether a set of given rationals, bases, and updated functions 
draw a closed shape in $D$ dimensional space, we first need to find the period of 
the update delta, $\Delta p_i$. We then know that the total update over a period will be repeated 
indefinitely. Consequently, we then seek to find the total change in position over a period. 
If the total change is $0$, the shape will close as the Turtle will end up at its starting point
after every period length. If the total update is nonzero, the Turtle will not draw a closed
shape.

\subsection{Finding period $T$}
\subsubsection{Finding the period of $\frac{n}{d}$}
We will first aim to find period $T$ of the state $\pmb{s}$.
For some $(n, b, d, \theta) \in \pmb{s}_0$, by \cite{MathOverflowFracPeriod},
we have that the period of the decimal expansion of $\frac{n}{d}$
can be determined by finding the smallest $\rationalPeriod$ such that
\begin{equation}
\label{eq:nd-period}
b ^ {\rationalPeriod} \equiv 1 \mod d.
\end{equation}
More generally though, any nontrivial $\rationalPeriod$ satisfying equation \ref{eq:nd-period}
will be a period of $\frac{n}{d}$.

Next, let $$
\rationalPeriodTotal = \lcm{j \in [k]}\; \rationalPeriod.
$$

\begin{remark}[Complexity]
\label{remark:periodcomplex}
  Interestingly, period finding of rational numbers is intimately tied to the discrete
  log problem and factoring. For more information, check out \cite{MathOverflowFracPeriod}.
  This gives some intuition that this closure problem may not be in BPP (Bounded Error Polynomial Time),
  but may be in BQP (Bounded Error Quantum Polynomial Time) by \cite{Shor_1997}.
\end{remark}

\subsubsection{Digital sum}
Next, we introduce the idea digital sums.
For some number $N \in \N$, $N$ can be represented in base $b$ via
\begin{equation}
  N = \sum_{i=0}^{m} d_i b^i
\end{equation}
where $m = \lceil \log_b N \rceil$ and, $\forall i \in [m]$, $d_i \in \Z_b$.
Then, we define function $\mathrm{digSum}: \N \rightarrow \Z_b$
to give the digital sum such that
\begin{equation}
  \mathrm{digSum}(N) = \sum_{i=0}^{m} d_i.
\end{equation}

Moreover, define $\digSumPeriodic \in \Z_b$
such that
% TODO: define T within the func from another func
% TODO: bettr defn
\begin{equation}
  \digSumPeriodic = \sum_{i = i_0}^{i_0 + \rationalPeriodTotal} \mathrm{digit}(n, d, b, i).
\end{equation}
In other words, $\rationalPeriodTotal$ is the digital sum over one period.

\begin{remark}[Complexity]
  For $d > 2$, prime, and coprime to $b$, we can find $\digSumPeriodic$
  in polytime by multiplying $(b - 1) \cdot \frac{d-1}{2} \mod b$ \cite{OnDecSeq}. The authors are unsure
  as to the complexity of finding $\digSumPeriodic$ otherwise.
\end{remark}

\subsubsection{Finding a period of $\rationalAngleNotime$} % TODO: redifine things in terms of theta...
For $(n^j, b^j, d^j, \theta_i^j) = s_i^j$,
recall that $\rationalAngle{i + 1} = \rationalAngle{i} + \mathrm{digit}(n, b, d, i) \mod b$.
So, after period $\rationalPeriodTotal$,
\begin{align*}
  \theta_{i + \rationalPeriodTotal} &= \left(\theta_{i} + \sum_{\ell = i}^{\rationalPeriodTotal + i} \mathrm{digit}(n, b, d, \ell)\right) \mod b\\
  &= \left(\theta_{i} +  \digSumPeriodic\right) \mod b.
\end{align*}
So, after $p$ periods of length $\rationalPeriodTotal$ where $p  \cdot \digSumPeriodic \equiv 0 \mod b$,
$$
\theta_{i + p\rationalPeriodTotal} \equiv \theta_i + 0 \equiv \theta_i.
$$
For simplicity, lets define
$$
  T^j = p\rationalPeriodTotal
$$
where $T^j$ is a period of the state for rational parameter $j$.
% TODO: remove $i$ from the state and somehow have it be implicit.... 

% Oh man... we are going to have to define an intermediate total period...
% One for all T'...
% and then have a sum for T
\subsubsection{Finding the period of $\pmb{s}$}
We can first see that for $\stateNotime{j} \in \pmb{s}$, $\stateNotime{j}$
has period of $\statePeriod$. So, $\pmb{s}$ must have a period, $\totalPeriod$, of
$$
\lcm{j \in [k]}\; T^j.
$$
I.e.\ $\pmb{s}_i = \pmb{s}_{i + T}$ for all $i \in \N$.




% TODO: cite https://math.stackexchange.com/questions/377683/length-of-period-of-decimal-expansion-of-a-fraction
% TODO: formula in poly time
% TODO: hmmm... specify i > n_0 (to get rid of initial stuff...)

% https://en.wikipedia.org/wiki/Digital_root#:~:text=The%20digital%20root%20(also%20repeated,single%2Ddigit%20number%20is%20reached.


\subsection{Finding the change in position over a period}
So now that we know the period of $\totalStateNotime$, we can ask if
$\Delta P_{i, i + T} = 0$. % TODO: reference

Note that 
$$
\Delta P_{i, i + T} = \Delta P_{q, q + T}
$$
for all $i, q \in \N$ by definition of periodicity.
So, we will drop the $i$ and replace it with a $0$.
Then,
\begin{align*}
  \Delta P_{0, T} &= \sum_{i = 1}^T \Delta p_i \\
  &= \sum_{i = 1}^T \combWithState{i} \\
  &= \sum_{i = 1}^T \left(\combSingleTerm{1}, ..., \combSingleTerm{D}\right) \\
  &= \left(\sum_{i = 1}^T \combSingleTerm{1}, ..., \sum_{i = 1}^T \combSingleTerm{D}\right).
\end{align*}
We can thus see that $\Delta P_{0, T} = \pmb{0} = (0, ..., 0)$ iff 
\begin{equation}
\label{eq:singleDelta0}
\sum_{i = 1}^T \combSingleTerm{d} = 0
\end{equation}
for all $d \in D$.
We can thus check for closure by computing \eqref{eq:singleDelta0} for each dimension.

\subsection{Algorithm complexity}
The algorithm we provide in equation \eqref{eq:singleDelta0} runs in time exponential 
in the size of the input assuming the Word RAM model. 
The period for the rational generated from rational parameter $j$, $1 \leq \rationalPeriod \leq d^j$. Then,
the period over all rationals generated from parameters is at most 
$$
  \lcm{j \in [k]} \rationalPeriod \leq \prod_{j \in [k]} \rationalPeriod \leq \left(\max_{j \in [k]} \; d^j \right)^k.
$$
Then, 
$0 \leq \totalPeriod \leq \rationalPeriodTotal \cdot \lcm{j \in [k]} b^j \leq \rationalPeriodTotal \left(\max_{j \in [k]} b^j\right)^k$.
And because evaluating the product in \eqref{eq:singleDelta0} takes $O(k)$ time,
we have that the time for \eqref{eq:singleDelta0} is at most
$$
  O\left(\left[\max_{j \in [k]} \; (b^j d^j)\right]^k \right).
$$

Then, note that computing the period of rational numbers via known classical methods takes 
exponential time in the number of digits of the denominator. So, computing
$\rationalPeriod$ takes $O(d^j)$ time. Then, computing the $\digSumPeriodic$
can take $O(d^j)$ time. 
We can thus see that period finding takes at most
$$
O\left(k \max_{j \in [k]} d^j \right)
$$
time.

Because \eqref{eq:singleDelta0} must be computed for each dimension, the algorithm
runs in 
$$
  O\left(k \max_{j \in [k]} d^j \right) + 
  O\left(\max_{j \in [k]} \; D (b^j d^j)^k \right) = 
  O\left(\max_{j \in [k]} \; D (b^j d^j)^k \right)
$$
time.
Note that $b^j, d^j$ are also exponential in the size of the input. 
We can thus see that our running time is quite atrocious (its worse than exponential).
Moreover, the algorithm does not produce a proof, verifiable in polytime, for closure 
or lack there of. Thus, our algorithm is in neither NP or coNP.

\section{Interesting Properties and an attempt at certificates}
We will now proceed to go over some interesting properties of the closure question
which may give rise to an algorithm in NP, coNP, or even BQP. These properties were discovered
in the author's pursuit of simplifying the question. Moreover, these properties may guide
some intuition as to the probability of closure for random rational seeds, a fixed $k$, 
and fixed bases $b$.

\subsection{Property 1: Restricted Monomials and Closure}
Define $A_d = \set{j \mid j \in [k] \; \text{and}\; \inclIndic{d}{j} = 1}$, in other words,
$A_d$ is the set of rational parameters which are included in determining the position along the $d$th dimension.
Also, for function $f: \Z_\commonBase^{|A_d|} \rightarrow \Z_\commonBase$ and $\{a_1, a_2, ..., a_{|A_d|} \} = A_d$, we will denote
$$
  f\left(\digSumPeriodicNoJ^{a_1}, \digSumPeriodicNoJ^{a_2}, ..., \digSumPeriodicNoJ^{a_{A_d}}\right)
  = f\left(\digSumPeriodicNoJ\right).
$$

Then, let
$$
\mathcal{M} = \{f : f(\pmb{x}) = \pm x_1 \pm x_2 ... \pm x_{|A_d|}\}.
$$
In other words, $\mathcal{M}$ is the set of all multinomials with $|A_d|$ variables with
degree 1 and coefficients $\pm1$.
Then,
if \begin{equation}
\label{eq:satisfy-multi}
f\left(\digSumPeriodicNoJ\right) \neq 0
\end{equation}
for all $f \in \mathcal{M}$ and $d \in [D]$, the turtle will always draw a closed shape. See 
Appendix \ref{appendix:prop1} for the proof.

Satisfying \eqref{eq:satisfy-multi} is true for all $f$ is equivalent to
$$
  \prod_{f \in \mathcal{M}} f(\digSumPeriodicNoJ) \neq 0.
$$

where $\prod_{f \in \mathcal{M}} f$ is a polynomial of degree
at most $2^{|A_d|} \leq 2^k$.

If we were to then assume that $(\digSumPeriodicNoJ^1, ..., \digSumPeriodicNoJ^k)$ 
is uniformly and randomly draw from $\Z_\commonBase^k$, we then know by the 
Schwartz-Zippel Lemma
$$
  \Pr\left[\prod_{f \in \mathcal{M}} f(\digSumPeriodicNoJ) = 0\right] < \frac{2^k}{\commonBase}.
$$
% TODO: cite,
So, this would leave us with
$$
  \Pr\left[\prod_{f \in \mathcal{M}} f(\digSumPeriodicNoJ) \neq 0\right] > 1 - \frac{2^k}{\commonBase}.
$$
In particular, this means that the probability of closure would be at least
$$
1 - \frac{2^k}{\commonBase}.
$$

Somewhat surprisingly, we can then see that probability of closure may increase exponentially
with a decreasing $k$. Moreover, a larger $\commonBase$ also increases the lower bound!

\begin{remark}[Randomness assumption]
  The randomness assumption, that $(\digSumPeriodicNoJ^1, ..., \digSumPeriodicNoJ^k)$
  is drawn from a random distribution is very much not true. But, given a rational parameter
  there does seem to be some element of randomness for $\digSumPeriodic$. See \cite{OnDecSeq} for more information.
\end{remark}
\section{Conclusion}

\section{Open Questions}


\section*{Acknowledgments}


\appendix
\section{Proving Property 1}
\label{appendix:prop1}
\appendix
\section{Proving Property 1 and 2}
\label{appendix:prop1}

First let $I = \sqrt{-1}$ instead of $i$. This is done as $i$ is already reserved
to represent the current time step.

Now, before getting to the main proof, we need to prove the following lemma
\begin{lemma}{For all $j \in [k]$ and $x, y \in \N$ where $y < \rationalPeriodTotal$, we have that 
  $$
    \theta_{xT' + y}^j = x\cdot \digSumPeriodic + \sum_{q = 0}^y \digit(n, b, d, q)
  $$}
  \label{lemma:angleBreakdown}
  \begin{proof}
    We can then see that for $(n, b, d, \theta_{x\rationalPeriodTotal + y}^j) \in \pmb{s}_{x\rationalPeriodTotal + y}$,
    \begin{align*}
      \rationalAngle{x \rationalPeriodTotal + y} &= \sum_{i = 0}^{x \rationalPeriodTotal + y} \mathrm{digit}(n, b, d, i) \\
      &= \sum_{p = 0}^{(x- 1)\rationalPeriodTotal} \sum_{q=0}^{\rationalPeriodTotal - 1} \mathrm{digit}(n, b, d, p\rationalPeriodTotal + q)
          + \sum_{q = x \rationalPeriodTotal}^{x \rationalPeriodTotal + y} \mathrm{digit}(n, b, d, q)\\
      &= x \cdot \digSumPeriodic + \sum_{q = x \rationalPeriodTotal}^{\rationalPeriodTotal + y}\mathrm{digit}(n, b, d, q)\\
      &= x \cdot \digSumPeriodic + \sum_{q = 0}^{y}\mathrm{digit}(n, b, d, q)
    \end{align*}
    because $\mathrm{digit}(n, b, d, x\rationalPeriodTotal + \ell) = \mathrm{digit}(n, b, d, \ell)$ for any $\ell \in \N$
    by definition of periodicity.
  \end{proof}
\end{lemma}

Let $\changeTotalOneDim$ be the change of position along dimension $d$
from timestep 0 to $T$.
We are now ready to determine if we ``close" along one dimension. I.e.\ does
$
\changeTotalOneDim = 0
$?

Define $A_d = \set{j \mid j \in [k] \; \text{and}\; \inclIndic{d}{j} = 1}$, in other words,
$A_d$ is the set of rational parameters which are included in determining the position along the $d$th dimension.
We can then see that
\begin{align*}
\changeTotalOneDim &=\sum_{i = 1}^T \combSingleTerm{d}\\
  &=\pm\sum_{i = 1}^T \prod_{j = 1}^k \left(\half \left(\exp\left({\numbToAngle \rationalAngle{i} I}\right) \pm \exp\left(-\numbToAngle \rationalAngle{i} I\right)\right)\right)^\inclIndic{d}{j}\\
  &= \pm 2^{-|A|} \sum_{p= 0}^{\frac{\totalPeriod}{\rationalPeriodTotal} - 1} \sum_{q = 0}^{\rationalPeriodTotal - 1}
    \prod_{j \in A_d} \left(
      \exp\left(\numbToAngle\rationalAngle{p \rationalPeriodTotal + q} I\right) \pm  \exp\left(-\numbToAngle\rationalAngle{p \rationalPeriodTotal + q} I\right)\right)
\end{align*}
by the Euler form of $\cos$ and $\sin$ and the fact that $\changeTotalOneDim$ is real.

\newcommand{\periodFrac}{\frac{\totalPeriod}{\rationalPeriod}}
\newcommand{\periodFracRational}{\frac{\rationalPeriodTotal}{\rationalPeriod}}

% TODO: not entirely correct with d
Next, observe that 
\begin{align*}
  &\prod_{j \in A_d} \left(
      \exp\left(\numbToAngle\rationalAngle{p \rationalPeriodTotal + q} I\right) \pm  \exp\left(-\numbToAngle\rationalAngle{p \rationalPeriodTotal + q} I\right)\right)\\
 =& 
  \exp(\numbToAngleNoJ{1}\rationalAngleJ{p \rationalPeriodTotal + q}{1} + \numbToAngleNoJ{2}\rationalAngleJ{p \rationalPeriodTotal + q}{2} + ... + \numbToAngleNoJ{d}\rationalAngleJ{p \rationalPeriodTotal + q}{d}) 
  \pm \exp(\numbToAngleNoJ{1}\rationalAngleJ{p \rationalPeriodTotal + q}{1} - \numbToAngleNoJ{2}\rationalAngleJ{p \rationalPeriodTotal + q}{2} + ... + \numbToAngleNoJ{d}\rationalAngleJ{p \rationalPeriodTotal + q}{d}) + ... \\
  &\pm \exp(-\numbToAngleNoJ{1}\rationalAngleJ{p \rationalPeriodTotal + q}{1} - \numbToAngleNoJ{2}\rationalAngleJ{p \rationalPeriodTotal + q}{2} - ... - \numbToAngleNoJ{d}\rationalAngleJ{p \rationalPeriodTotal + q}{d})
\end{align*}
which then equals
\begin{equation}
 \sum_{\beta \in \set{0,1}^{|A_d|}} 
		\pm
    \exp\left(
      	\commonToAngle I
        \sum_{j \in A_d} -1 ^ {\beta_{(j)}}
          \numbToCommon \rationalAngle{p \rationalPeriodTotal + q}
      \right)
\end{equation}
where $\beta$ can be though of as a bit string deciding whether the angle from seed
$j \in A_d$ is added to or subtracted from the exponent.

Then, we have that
% TODO: update with common base stuff...
\begin{align*}
 \changeTotalOneDim =&
  \pm 2^{-|A|} \sum_{p= 0}^{\frac{\totalPeriod}{\rationalPeriodTotal} - 1} \sum_{q = 0}^{\rationalPeriodTotal - 1}
 \sum_{\beta \in \set{0,1}^{|A_d|}} 
 	 \pm
    \exp\left(
      	\commonToAngle I
        \sum_{j \in A_d} -1 ^ {\beta_{(j)}}
        \numbToCommon \rationalAngle{p \rationalPeriodTotal + q}
      \right)\\
 =&
  \pm 2^{-|A|} 
 \sum_{\beta \in \set{0,1}^{|A_d|}} 
		\pm
     \sum_{p= 0}^{\frac{\totalPeriod}{\rationalPeriodTotal} - 1} \sum_{q = 0}^{\rationalPeriodTotal - 1}
    \exp\left(
      	\commonToAngle I
        \sum_{j \in A_d} -1 ^ {\beta_{(j)}}
        \numbToCommon \rationalAngle{p \rationalPeriodTotal + q}
      \right).
\end{align*}

\newcommand{\eqWTSInnerProd}{
  \sum_{p= 0}^{\frac{\totalPeriod}{\rationalPeriodTotal} - 1} \sum_{q = 0}^{\rationalPeriodTotal - 1}
    \exp\left(
      	\commonToAngle I 
        \sum_{j \in A_d} -1 ^ {\beta_{(j)}}
        \numbToCommon 
        \rationalAngle{p \rationalPeriodTotal + q}I
      \right)}

Then, lets fix some $\beta \in \set{0, 1}^{|A_d|}$, define $Q$ such that
\begin{equation}
\label{eq:wts01}  
  Q = \eqWTSInnerProd.
\end{equation}
We will proceed $Q$ to show 2 distinct cases where $Q = 0$ for any choice of $\beta$.

Observe that 
\begin{align}
  \exp(\rationalAngle{p \rationalPeriodTotal + q}I)
  &= \exp\left(
      p \cdot \digSumPeriodic +
      \sum_{\ell = p \rationalPeriodTotal}^{p \rationalPeriodTotal + q}\mathrm{digit}(n^j, b^j, d^j, \ell)\right)\tag{by lemma \ref{lemma:angleBreakdown}} \\
      \label{eq:expBreakdown}
      &= \exp\left(p \cdot \digSumPeriodic\right) \exp\left(\sum_{\ell = 0}^{q}\mathrm{digit}(n, b, d, \ell)\right).
\end{align}

So then, by equation \eqref{eq:expBreakdown}, we get that
\begin{align}
&\exp\left(
    \sum_{j \in A_d} -1 ^ {\beta_{(j)}}
    \cdot
    \numbToCommon \cdot
    \rationalAngle{p \rationalPeriodTotal + q}I
\right)\notag \\ 
\label{eq:decomposeExp}
=&
\exp\left(
    I \sum_{j \in A_d} -1 ^ {\beta_{(j)}}
    \cdot p \cdot
    \digSumPeriodic
\right)
\exp\left(
    I \sum_{j \in A_d} -1 ^ {\beta_{(j)}} \numbToCommon
      \sum_{\ell = 0}^{q}\mathrm{digit}(n^j, b^j, d^j, \ell)
\right).
\end{align}

% TODO: careful read through and w/ I.
% TODO: clean up and fix the lack of $2pi$ correction factor

We then use \eqref{eq:decomposeExp} to show that $Q$ equals
\begin{equation}
\label{eq:inner-outer-split}
  \sum_{p= 0}^{\frac{\totalPeriod}{\rationalPeriodTotal} - 1} \left[
    I \exp\left(
      pI \commonToAngle \sum_{j \in A_d} -1 ^ {\beta_{(j)}}
      \digSumPeriodic
    \right)
  \left(
  I \sum_{q = 0}^{\rationalPeriodTotal - 1}
    \exp\left(
        \commonToAngle
        \sum_{j \in A_d} -1 ^ {\beta_{(j)}} \numbToCommon
          \sum_{\ell = 0}^{q}\mathrm{digit}(n^j, b^j, d^j, \ell)
    \right)\right)\right].
\end{equation}

\subsubsection*{Case 1: $\sum_{j \in A_d} (-1) ^ {\beta_{(j)}} \digSumPeriodic \not\equiv 0 \mod \commonBase$}
Define
$$
  C_\beta = \sum_{q = 0}^{\rationalPeriodTotal - 1}
  \exp\left(
      \commonToAngle
      \sum_{j \in A_d} -1 ^ {\beta_{(j)}} \numbToCommon
        \sum_{\ell = 0}^{q}\mathrm{digit}(n, b, d, \ell)
  \right).
$$
Moreover, note that 
\begin{align*}
  &\exp\left(
      pI \commonToAngle \sum_{j \in A_d} -1 ^ {\beta_{(j)}}
      \digSumPeriodic
    \right)
  = \prod_{j \in A_d} \exp\left(
      -1^{\beta_{(j)}}
     \cdot
    	\commonToAngle
     \cdot
      pI \cdot \digSumPeriodic 
    \right)
\end{align*}
and then  because $-1^{\beta_{(j)}} \cdot \commonToAngle pI \cdot \digSumPeriodic \equiv 0 \mod \commonBase$,
\begin{equation*}
   -1^{\beta_{(j)}} \cdot \commonToAngle pI \cdot \digSumPeriodic = \alpha \cdot 2\pi
\end{equation*}
for some $\alpha \in \N$. So,
\begin{equation*}
  \exp\left(-1^{\beta_{(j)}} \cdot\commonToAngle \cdot pI \cdot \digSumPeriodic \right) = \exp\left(0\right) = 1
\end{equation*}
when $p = \frac{\totalPeriod}{\rationalPeriodTotal}$. 


We can conclude that
\begin{equation*}
 \exp\left(
      I \commonToAngle \sum_{j \in A_d} -1 ^ {\beta_{(j)}}
      \digSumPeriodic
    \right) 
\end{equation*}
is a $\totalOverRationalFrac^{th}$ root of unity iff
$$\sum_{j \in A_d} -1 ^ {\beta_{(j)}}
      \digSumPeriodic \not\equiv 0 \mod \commonBase
$$

So, for $\sum_{j \in A_d} -1 ^ {\beta_{(j)}} \digSumPeriodic \not\equiv 0$, we have that
\begin{align*}
  \eqWTSInnerProd &= C_\beta \sum_{p= 0}^{\frac{\totalPeriod}{\rationalPeriodTotal} - 1}
    \exp\left(
      pI \commonToAngle \sum_{j \in A_d} -1 ^ {\beta_{(j)}}
      \digSumPeriodic
    \right) \\
    &= C_\beta \sum_{p = 0}^{\totalOverRationalFrac - 1}\exp\left(W_{\totalOverRationalFrac}^p\right) \\
    &= 0.
\end{align*}
where $W_{\totalOverRationalFrac}^p$ is the $\totalOverRationalFrac^{th}$ root of unity.

\subsubsection*{Case 2: $\sum_{j \in A_d} -1 ^ {\beta_{(j)}} \digSumPeriodic \equiv 0 \mod \commonBase$}
Now,
\begin{align*}
  \eqWTSInnerProd &= C_\beta \sum_{p= 0}^{\frac{\totalPeriod}{\rationalPeriodTotal} - 1} \exp(0) \\
  &= C_\beta.
\end{align*}
So $$\eqWTSInnerProd = 0$$ if $C_\beta = 0$.

\subsubsection*{To conclude}
If, $\forall \beta \in \set{0, 1}^{|A_d|}$, $\sum_{j \in A_d} -1 ^ {\beta_{(j)}} \digSumPeriodic \not\equiv 0$
or $C_\beta = 0$, then
\begin{align*}
  \changeTotalOneDim &= \sum_{i = 1}^T \combSingleTerm{d} \\&=  2^{-|A|}
  \sum_{\beta \in \set{0,1}^{|A_d|}} 
      \sum_{p= 0}^{\frac{\totalPeriod}{\rationalPeriodTotal} - 1} \sum_{q = 0}^{\rationalPeriodTotal - 1}
     \exp\left(
       \commonToAngle
         \sum_{j \in A_d} -1 ^ {\beta_{(j)}}
         \numbToCommon
         \rationalAngle{p \rationalPeriodTotal + q}I
       \right) \\
        &= 0.
\end{align*}
If the above is true for all $d \in D$, then 
$\Delta P_{0, T} = 0$.


\bibliographystyle{alpha}
\bibliography{bib/ref}



\end{document}

% TODO: put b\theta / 2pi in State UPDATE