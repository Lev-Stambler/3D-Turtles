\documentclass[11pt,titlepage]{article}
\usepackage{TCSToolkit}

\newcommand{\dofloor}[1]{\lfloor#1\rfloor}
\newcommand{\doceil}[1]{\lceil#1\rceil}

%%%%% Stuff you can change %%%%%%%%%%%%%%%%%%%%%%%%%%%%%%%%%%
\newcommand{\myname}{Lev Stambler}
%

%% Custom Commands
\newcommand{\rationalPeriod}{{T^j}'}
\newcommand{\rationalPeriodTotal}{{T}'}
\newcommand{\statePeriod}{T^j}
\newcommand{\seedAngleJ}[2]{{\theta^{#2}_{#1}}}
\newcommand{\seedAngle}[1]{{\theta^j_{#1}}}
\newcommand{\seedAngleNotime}{{\theta}^j}
\newcommand{\state}[2]{{s_{#1}}^{#2}}
\newcommand{\stateNotime}[1]{s^{#1}}
\newcommand{\totalPeriod}{T}
\newcommand{\totalState}{\pmb{s}_i}
\newcommand{\totalStateNotime}{\pmb{s}}
\newcommand{\comb}{Comb}
\newcommand{\combWithState}[1]{Comb\left(s^1_{#1}, s^2_{#1}, ..., s^k_{#1}\right)}
\newcommand{\inclIndic}[2]{{\mathrm{incl}_{#1}^{#2}}}
\newcommand{\combSingleTerm}[1]{\prod_{j=1}^k \cos(\theta_j) ^ {\inclIndic{#1}{j}}}
\newcommand{\digSumPeriodic}{\mathrm{digSumPeriodic}^j}


%
% Final tip: you can reference HW5 in your TeX using \Cref{hw:5}; 
% and, you can reference HW5.Problem3 in your TeX using \Cref{prob:5.3}
%


%%%%% Section-renaming code by egreg
\makeatletter
% we use \prefix@<level> only if it is defined
\renewcommand{\@seccntformat}[1]{%
  \ifcsname prefix@#1\endcsname
    \csname prefix@#1\endcsname
  \else
    \csname the#1\endcsname\quad
  \fi}
% Now we define our homework section prefixes
% \newcommand\prefix@section{Homework \thesection: }
% \newcommand{\prefix@subsection}{Problem \thesubsection: }
% \newcommand{\prefix@subsubsection}{Section \thesubsubsection: }
\makeatother
%%%%%




\begin{document}

\title{Drawing Math}

\author{\myname}

\date{\today}

\maketitle

\pagebreak
\section{Introduction}
- Numberphile video
- I was bored and this looked fun
- I wasn't paying attention in one of my lectures...
- This paper is purely for funs... for now

\section{Background}
- Digital math
- Euler's formula
- I payed attention in some of my lectures

\section{Definitions and questions}
\subsection{Definitions}
Say you (yes you!) had a turtle living in $D$ dimensional Euclidean
space and in discrete time. At time step $i$, where $i \in \Z$ and $i > 0$,
the turtle has position $p_i \in \R^D$.
Then, lets define $\Delta p_{i+1} = p_{i+1} - p_i$; in other words, $\Delta p_{i+1}$ is the change in position from time $i$ to $i + 1$.

Now say that the turtle's movement is determined by $k$ seed parameters drawn from
the same set. Then, for some state space $\mathcal{S}$,
define $s_i^j \in \mathcal{S}$ to be some arbitrary
state associated with timestamp $i$ for the $j$th seed parameter where $j \in [k]$.
Also, define $\pmb{s}_i = (s_i^1, s_i^2, ..., s_i^k)$.
Next we will define a set of functions $SU^j: \mathcal{S} \rightarrow \mathcal{S}$
(for $S$tate $U$pdater) such that $s_{i + 1}^j = SU^j(s_i^j, i)$. Note that for
$j, a \in [m]$ where $j \neq a$, $s_{i+1}^j$ is determined solely by $s_i^j$ and $i$ and not
$s_i^a$.

Now that we have our machinery built up, lets define $Comb: \mathcal{S}^k \rightarrow \R^d$ such that
$$
  \Delta p_{i + 1} = \combWithState{i + 1}.
$$
In other words, $Comb$ takes in the state of each seed and
returns an update to the position of the turtle.

Finally, let us define
$$
  \Delta P_{a, b} = \sum_{i = a}^{b} \Delta p_i.
$$
In other words, $\Delta P_{a, b}$ is the change in position from timestep $a$ to $b$.

\subsection{The problem}
Say we are given, $Comb$, hh$SU^j$, $p_0$, and $s^j_0$ for all $j \in [k]$.
Informally, the question is whether the turtle draws a ``closed" shape or not.\\
More formally, is there some period $T$ such that 
$$
  p_{i + \ell T} = p_{i}
$$
for $i, \ell \in \N$.
Then, note that if there exists a period $T$ such that 
$\Delta P_{i, i + \ell T} = 0$ for all $i, \ell \in N$, 
$p_{i + \ell T} = p_{i}$ and the turtle forms a closed shape.

% TODO: ask closure question
% TODO: show result that if $(s_{a}^1, ... ) = (s_{b}^1,...)$ then $b-a$ is a period for the state
% and Delta p_a = p_b and is thus invariant under mod b-a. So, if Delta p_{b-a} \neq 0 
% we have that the turtle will not move in a closed shape. If Delta p_{b-a} = 0, it will  

\subsection{Specifying the task ahead of us}
For our case, we consider $Comb, SU^j$ to all be memoryless (i.e.\ there output
is uniquely determined by the current input). So, we can simplify the overall question.
If, $\pmb{s}_i = \pmb{s}_{i + \ell T}$ for some $T \in \N$ and all $i \in \N$, then 
$\Delta p_{i} = \Delta p_{i + \ell T}$. So then,
$\Delta P_{i, i + \ell T} = \Delta P_{i, i + \ell' T}$ for all $\ell, \ell' \in \N$.
Thus, $T$ is a period of the change in position. We can thus
break down our problem into two parts:
\begin{enumerate}
  \item Finding the period, $T$, of the state $\pmb{s}$.
  \item Checking whether $\Delta P_{i, i + T} = 0$.
\end{enumerate}

\subsection{Some more restrictions on our problem}
We further restrict the problem by only considering the case where
$$
  Comb((., ., ., ., \theta^1), (., ., ., ., \theta^2), ..., (., ., ., ., \theta^k)) =
    \left(\combSingleTerm{1}, ..., \combSingleTerm{D}\right)
$$
where $\mathrm{incl}_d^j \in \set{0, 1}$ for $d \in [D]$ indicates whether to include a given
$x \in R$ determined by seed $j$ for position update in the $d$th dimension.

We will also let $\mathcal{S} = N^5$ where for $(n, d, b, \theta) \in \mathcal{S}$,
$n$ is the numerator of a rational in fraction form, $d$ is the denominator, 
$b$ is the base (i.e\. base 10, base 12, etc.), $i$ is the current time step,
and $\frac{2\theta}{b \pi}$ is an ``angle" associated with the state.
Moreover, we will only consider
$$
  SU^j(n, b, d, \theta) = (n, b, d, \theta + \mathrm{digit}(n, b, d, i) \mod b).
$$

where $\mathrm{digit}(n, b, d, i)$ gives us the $i$th digit of the decimal expansion of $\frac{n}{d}$
in base $b$.

Finally, for simplicity's sake, assume that $\theta = 0$ for all $(n, b, d, \theta) \in \pmb{s}_0$,
$n < d$, and $\frac{n}{d}$'s decimal expansion is periodic after some $N \geq 0$ decimal places
and does not terminate in base $b$.

\subsubsection{Some intuition}
While the restrictions may seem arbitrary, they aptly match our original problem statement.
Take the three dimensional case with two seeds. The two seeds can be thought
of as the fraction which generates the updates to two angles, $\theta, \alpha$.
$\theta$ and $\alpha$ can then uniquely determine a direction in 3D space in which the 
turtle is pointing. 
So, at each time step $i + 1$, $\theta_{i + 1}$ equals to $\theta_i$, but rotated by
some fraction of $2 \pi$ determined by the $i$th digit of the decimal expansion of one fraction and its base.
% TODO: clear up.

% TODO: put your own diagram in!!


\section{Results}
\subsection{Finding period $T$}
\subsubsection{Finding the period of $\frac{n}{d}$}
We will first aim to find period $T$ of the state $\pmb{s}$.
For some $(n, b, d, \theta) \in \pmb{s}_0$, by \cite{MathOverflowFracPeriod},
we have that the period of the decimal expansion of $\frac{n}{d}$
can be determined by finding the smallest $\rationalPeriod$ such that
\begin{equation}
\label{eq:nd-period}
b ^ {\rationalPeriod} \equiv 1 \mod d.
\end{equation}
See, appendix \ref{AppendixPeriod} for more detail.
More generally though, any nontrivial $\rationalPeriod$ satisfying equation \ref{eq:nd-period}
will be a period of $\frac{n}{d}$.

Next, let $$
\rationalPeriodTotal = \lcm{j \in [k]}\; \rationalPeriod.
$$

\begin{remark}[Polytime]
  TODO: remark about polytime
\end{remark}

\subsubsection{Digital sum}
Next, we introduce the idea digital sums.
For some number $N \in \N$, $N$ can be represented in base $b$ via
\begin{equation}
  N = \sum_{i=0}^{m} d_i b^i
\end{equation}
where $m = \lceil \log_b N \rceil$ and, $\forall i \in [m]$, $d_i \in \Z_b$.
Then, we define function $\mathrm{digSum}: \N \rightarrow \Z_b$
to give the digital sum such that
\begin{equation}
  \mathrm{digSum}(N) = \sum_{i=0}^{m} d_i.
\end{equation}

Moreover, define function $\digSumPeriodic \in \Z_b$
such that
% TODO: define T within the func from another func
% TODO: bettr defn
\begin{equation}
  \digSumPeriodic = \sum_{i = i_0}^{i_0 + \rationalPeriodTotal} \mathrm{digit}(n, d, b, i)
\end{equation}

\begin{remark}[Finding the digit sum of a rational]
  aaaa
\end{remark}

\subsubsection{Finding a period of $\seedAngleNotime$} % TODO: redifine things in terms of theta...
For $(n^j, b^j, d^j, \theta_i^j) = s_i^j$,
recall that $\seedAngle{i + 1} = \seedAngle{i} + \mathrm{digit}(n, b, d, i) \mod b$.
So, after period $\rationalPeriodTotal$,
\begin{align*}
  \theta_{i + \rationalPeriodTotal} &= \left(\theta_{i} + \sum_{\ell = i}^{\rationalPeriodTotal + i} \mathrm{digit}(n, b, d, \ell)\right) \mod b\\
  &= \left(\theta_{i} +  \digSumPeriodic\right) \mod b\\
  &= \theta_{i} \mod b +  \digSumPeriodic \mod b.
\end{align*}
So, after $p$ periods of length $\rationalPeriodTotal$ where $p  \cdot \digSumPeriodic \equiv 0 \mod b$,
$$
\theta_{i + p\rationalPeriodTotal} = \theta_i.
$$
For simplicity, lets define
$$
  T^j = p\rationalPeriodTotal
$$
where $T^j$ is a period of the state for seed $j$.
% TODO: remove $i$ from the state and somehow have it be implicit.... 

% Oh man... we are going to have to define an intermediate total period...
% One for all T'...
% and then have a sum for T
\subsubsection{Finding the period of $\pmb{s}$}
We can first see that for $\stateNotime{j} \in \pmb{s}$, $\stateNotime{j}$
has period of $\statePeriod$. So, $\pmb{s}$ must have a period, $\totalPeriod$, of
$$
\lcm{j \in [k]}\; T^j.
$$
I.e.\ $\pmb{s}_i = \pmb{s}_{i + T}$ for all $i \in \N$.




% TODO: cite https://math.stackexchange.com/questions/377683/length-of-period-of-decimal-expansion-of-a-fraction
% TODO: formula in poly time
% TODO: hmmm... specify i > n_0 (to get rid of initial stuff...)

% https://en.wikipedia.org/wiki/Digital_root#:~:text=The%20digital%20root%20(also%20repeated,single%2Ddigit%20number%20is%20reached.


\subsection{Does it close? $\Delta P_{i, i + T} = 0$?}
So now that we know the period of $\totalStateNotime$, we can ask if
$\Delta P_{i, i + T} = 0$. % TODO: reference

Note that 
$$
\Delta P_{i, i + T} = \Delta P_{q, q + T}
$$
for all $i, q \in \N$ by definition of periodicity.
So, we will drop the $i$ and replace it with a $0$.
Then,
\begin{align*}
  \Delta P_{0, T} &= \sum_{i = 1}^T \Delta p_i \\
  &= \sum_{i = 1}^T \combWithState{i} \\
  &= \sum_{i = 1}^T \left(\combSingleTerm{1}, ..., \combSingleTerm{D}\right) \\
  &= \left(\sum_{i = 1}^T \combSingleTerm{1}, ..., \sum_{i = 1}^T \combSingleTerm{D}\right).
\end{align*}
We can thus see that $\Delta P_{0, T} = \pmb{0} = (0, ..., 0)$ iff 
\begin{equation}
\label{eq:singleDelta0}
\sum_{i = 1}^T \combSingleTerm{d} = 0
\end{equation}
for all $d \in D$.

\subsubsection{Closing along a single dimension}
Finally, to see if a shape will close,
we just need to find if equation \eqref{eq:singleDelta0} holds true for all dimensions.
First, let $A_d = \set{j \mid j \in [k] \; ; \inclIndic{d}{j} = 1}$
We can then see that
\begin{align*}
&\sum_{i = 1}^T \combSingleTerm{d}\\
&=\sum_{i = 1}^T \prod_{j = 1}^k \left(\half \left(e^{\seedAngle{i} I} +  e^{-\seedAngle{i} I}\right)\right)^\inclIndic{d}{j}\\
&=2^{-|A|} \sum_{i = 1}^T \prod_{j \in A_d} \left(e^{\seedAngle{i} I} +  e^{-\seedAngle{i} I}\right)\\
\end{align*}

\newcommand{\periodFrac}{\frac{\totalPeriod}{\rationalPeriod}}
\newcommand{\periodFracRational}{\frac{\rationalPeriodTotal}{\rationalPeriod}}

We can then see that for $(n, b, d, \theta_i) \in \pmb{s}_i$ and $x, y \in \N$ where $y < \rationalPeriodTotal$,
\begin{align}
  \seedAngle{x \rationalPeriodTotal + y} &= \sum_{i = 0}^{x \rationalPeriodTotal + y} \mathrm{digit}(n, b, d, i) \notag \\
  &= \sum_{p = 0}^{(x- 1)\rationalPeriodTotal} \sum_{q=0}^{\rationalPeriodTotal - 1} \mathrm{digit}(n, b, d, p\rationalPeriodTotal + q)
    + \sum_{p = x \rationalPeriodTotal}^{x \rationalPeriodTotal + y} \mathrm{digit}(n, b, d, p) \notag \\
    % ARGS::....
  \label{eq:angleBreakdown}
  &= x \cdot \digSumPeriodic + \sum_{p = x \rationalPeriodTotal}^{\rationalPeriodTotal + y}\mathrm{digit}(n, b, d, p) \notag \\
  &= x \cdot \digSumPeriodic + \sum_{p = 0}^{y}\mathrm{digit}(n, b, d, p)
\end{align}
because $\mathrm{digit}(n, b, d, x\rationalPeriodTotal + \ell) = \mathrm{digit}(n, b, d, \ell)$ for any $\ell \in \N$
by definition of periodicity.


Observe that 
\begin{align*}
 &\prod_{j \in A_d} \left(e^{\seedAngle{p \rationalPeriodTotal + q} I} +  e^{-\seedAngle{p \rationalPeriodTotal + q} I}\right) \\
 =& \exp(\seedAngleJ{p \rationalPeriodTotal + q}{1} + \seedAngleJ{p \rationalPeriodTotal + q}{2} + ... + \seedAngleJ{p \rationalPeriodTotal + q}{d}) 
  + \exp(\seedAngleJ{p \rationalPeriodTotal + q}{1} - \seedAngleJ{p \rationalPeriodTotal + q}{2} + ... + \seedAngleJ{p \rationalPeriodTotal + q}{d}) + ... \\
  &+ \exp(\seedAngleJ{p \rationalPeriodTotal + q}{1} - \seedAngleJ{p \rationalPeriodTotal + q}{2} - ... - \seedAngleJ{p \rationalPeriodTotal + q}{d})
\end{align*}
which then equals
\begin{equation}
 \sum_{\beta \in \set{0,1}^{|A_d|}} 
    \exp\left(
        \sum_{j \in A_d} -1 ^ {\beta_{(j)}}
        \seedAngle{p \rationalPeriodTotal + q}I
      \right).
\end{equation}

Then, 
\begin{align*}
 &2^{-|A|} \sum_{i = 1}^T \prod_{j \in A_d} \left(e^{\seedAngle{i} I} +  e^{-\seedAngle{i} I}\right)\\
 =& 2^{-|A|} \sum_{p= 0}^{\frac{\totalPeriod}{\rationalPeriodTotal} - 1} \sum_{q = 0}^{\rationalPeriodTotal - 1}\prod_{j \in A_d} \left(e^{\seedAngle{p \rationalPeriodTotal + q} I} +  e^{-\seedAngle{p \rationalPeriodTotal + q} I}\right)\\
 =&
  2^{-|A|} \sum_{p= 0}^{\frac{\totalPeriod}{\rationalPeriodTotal} - 1} \sum_{q = 0}^{\rationalPeriodTotal - 1}
 \sum_{\beta \in \set{0,1}^{|A_d|}} 
    \exp\left(
        \sum_{j \in A_d} -1 ^ {\beta_{(j)}}
        \seedAngle{p \rationalPeriodTotal + q}I
      \right)\\
 =&
  2^{-|A|} 
 \sum_{\beta \in \set{0,1}^{|A_d|}} 
     \sum_{p= 0}^{\frac{\totalPeriod}{\rationalPeriodTotal} - 1} \sum_{q = 0}^{\rationalPeriodTotal - 1}
    \exp\left(
        \sum_{j \in A_d} -1 ^ {\beta_{(j)}}
        \seedAngle{p \rationalPeriodTotal + q}I
      \right).
\end{align*}

Then, lets fix some $\beta \in \set{0, 1}^{|A_d|}$, we will now show that 
\begin{equation}
\label{eq:wts01}  
  \sum_{p= 0}^{\frac{\totalPeriod}{\rationalPeriodTotal} - 1} \sum_{q = 0}^{\rationalPeriodTotal - 1}
    \exp\left(
        \sum_{j \in A_d} -1 ^ {\beta_{(j)}}
        \seedAngle{p \rationalPeriodTotal + q}I
      \right)
\end{equation}
always equals $0$ or $1$ for any choice of $\beta$.

Observe that 
\begin{align}
  \exp(\seedAngle{p \rationalPeriodTotal + q}I) &= \exp\left(p \cdot \digSumPeriodic +
    \sum_{\ell = p \rationalPeriodTotal}^{p \rationalPeriodTotal + q}\mathrm{digit}(n, b, d, \ell)\right) \tag{by equation \eqref{eq:angleBreakdown}} \\
    \label{eq:expBreakdown}
    &= \exp\left(p \cdot \digSumPeriodic\right) \exp\left(\sum_{\ell = 0}^{q}\mathrm{digit}(n, b, d, \ell)\right).
\end{align}

So then, by equation \eqref{eq:expBreakdown}, we get that
\begin{align}
&\exp\left(
    \sum_{j \in A_d} -1 ^ {\beta_{(j)}}
    \seedAngle{p \rationalPeriodTotal + q}I = 
\right)\notag \\ 
\label{eq:decomposeExp}
=&
\exp\left(
    \sum_{j \in A_d} -1 ^ {\beta_{(j)}}
    \cdot p \cdot
    \digSumPeriodic
\right)
\exp\left(
    \sum_{j \in A_d} -1 ^ {\beta_{(j)}}
      \sum_{\ell = 0}^{q}\mathrm{digit}(n, b, d, \ell).
\right).
\end{align}

We then use \eqref{eq:decomposeExp} to show that \eqref{eq:wts01} equals
\begin{equation*}
  \sum_{p= 0}^{\frac{\totalPeriod}{\rationalPeriodTotal} - 1} \left[
    \exp\left(
      pI \sum_{j \in A_d} -1 ^ {\beta_{(j)}}
      \digSumPeriodic
    \right)
  \left(
  \sum_{q = 0}^{\rationalPeriodTotal - 1}
    \exp\left(
        \sum_{j \in A_d} -1 ^ {\beta_{(j)}}
          \sum_{\ell = 0}^{q}\mathrm{digit}(n, b, d, \ell).
    \right)\right)\right].
\end{equation*}
% TODO: second part can be thought of as a constant for now... ie C2=
% Then, we can think of the inner first sum as a constant of sorts, (C1)
% And we know that pI(T/T') = 1 and that pI0 = 1. So, if C1 \neq 0, we get some roots of unity
% and then the whole sum is 0...
% If it does equal 0, you have to check if C2 = 0... don't really know how to do this...
% If C1 = 0 and C2 not equal to 0, then 

% TODO: then think of notation for the "distributed" polynomial, then you can expand...
% / assignment of +- strings... oy vey

% TODO: no clue what is going on with C2... (maybe throw in some pictures...)

% TODO: overall goal, show this is in coNP...

% A bunch of sums and multiplies
% \cite{watrous2018theory}

\section{Conclusion}

\section{Open Questions}


\section*{Acknowledgments}


\appendix
\section{Finding the period of $\frac{n}{d}$}
\label{AppendixPeriod}
aaaaa

\bibliographystyle{alpha}
\bibliography{bib/ref}



\end{document}

% TODO: put b\theta / 2pi in State UPDATE