\appendix
\section{Proving Property 1 and 2}
\label{appendix:prop1}

First let $I = \sqrt{-1}$ instead of $i$. This is done as $i$ is already reserved
to represent the current time step.

Now, before getting to the main proof, we need to prove the following lemma
\begin{lemma}{For all $j \in [k]$ and $x, y \in \N$ where $y < \rationalPeriodTotal$, we have that 
  $$
    \theta_{xT' + y}^j = x\cdot \digSumPeriodic + \sum_{q = 0}^y \digit(n, b, d, q)
  $$}
  \label{lemma:angleBreakdown}
  \begin{proof}
    We can then see that for $(n, b, d, \theta_{x\rationalPeriodTotal + y}^j) \in \pmb{s}_{x\rationalPeriodTotal + y}$,
    \begin{align*}
      \rationalAngle{x \rationalPeriodTotal + y} &= \sum_{i = 0}^{x \rationalPeriodTotal + y} \mathrm{digit}(n, b, d, i) \\
      &= \sum_{p = 0}^{(x- 1)\rationalPeriodTotal} \sum_{q=0}^{\rationalPeriodTotal - 1} \mathrm{digit}(n, b, d, p\rationalPeriodTotal + q)
          + \sum_{q = x \rationalPeriodTotal}^{x \rationalPeriodTotal + y} \mathrm{digit}(n, b, d, q)\\
      &= x \cdot \digSumPeriodic + \sum_{q = x \rationalPeriodTotal}^{\rationalPeriodTotal + y}\mathrm{digit}(n, b, d, q)\\
      &= x \cdot \digSumPeriodic + \sum_{q = 0}^{y}\mathrm{digit}(n, b, d, q)
    \end{align*}
    because $\mathrm{digit}(n, b, d, x\rationalPeriodTotal + \ell) = \mathrm{digit}(n, b, d, \ell)$ for any $\ell \in \N$
    by definition of periodicity.
  \end{proof}
\end{lemma}

Let $\changeTotalOneDim$ be the change of position along dimension $d$
from timestep 0 to $T$.
We are now ready to determine if we ``close" along one dimension. I.e.\ does
$
\changeTotalOneDim = 0
$?

Define $A_d = \set{j \mid j \in [k] \; \text{and}\; \inclIndic{d}{j} = 1}$, in other words,
$A_d$ is the set of rational parameters which are included in determining the position along the $d$th dimension.
We can then see that
\begin{align*}
\changeTotalOneDim &=\sum_{i = 1}^T \combSingleTerm{d}\\
  &=\pm\sum_{i = 1}^T \prod_{j = 1}^k \left(\half \left(\exp\left({\numbToAngle \rationalAngle{i} I}\right) \pm \exp\left(-\numbToAngle \rationalAngle{i} I\right)\right)\right)^\inclIndic{d}{j}\\
  &= \pm 2^{-|A|} \sum_{p= 0}^{\frac{\totalPeriod}{\rationalPeriodTotal} - 1} \sum_{q = 0}^{\rationalPeriodTotal - 1}
    \prod_{j \in A_d} \left(
      \exp\left(\numbToAngle\rationalAngle{p \rationalPeriodTotal + q} I\right) \pm  \exp\left(-\numbToAngle\rationalAngle{p \rationalPeriodTotal + q} I\right)\right)
\end{align*}
by the Euler form of $\cos$ and $\sin$ and the fact that $\changeTotalOneDim$ is real.

\newcommand{\periodFrac}{\frac{\totalPeriod}{\rationalPeriod}}
\newcommand{\periodFracRational}{\frac{\rationalPeriodTotal}{\rationalPeriod}}

% TODO: not entirely correct with d
Next, observe that 
\begin{align*}
  &\prod_{j \in A_d} \left(
      \exp\left(\numbToAngle\rationalAngle{p \rationalPeriodTotal + q} I\right) \pm  \exp\left(-\numbToAngle\rationalAngle{p \rationalPeriodTotal + q} I\right)\right)\\
 =& 
  \exp(\numbToAngleNoJ{1}\rationalAngleJ{p \rationalPeriodTotal + q}{1} + \numbToAngleNoJ{2}\rationalAngleJ{p \rationalPeriodTotal + q}{2} + ... + \numbToAngleNoJ{d}\rationalAngleJ{p \rationalPeriodTotal + q}{d}) 
  \pm \exp(\numbToAngleNoJ{1}\rationalAngleJ{p \rationalPeriodTotal + q}{1} - \numbToAngleNoJ{2}\rationalAngleJ{p \rationalPeriodTotal + q}{2} + ... + \numbToAngleNoJ{d}\rationalAngleJ{p \rationalPeriodTotal + q}{d}) + ... \\
  &\pm \exp(-\numbToAngleNoJ{1}\rationalAngleJ{p \rationalPeriodTotal + q}{1} - \numbToAngleNoJ{2}\rationalAngleJ{p \rationalPeriodTotal + q}{2} - ... - \numbToAngleNoJ{d}\rationalAngleJ{p \rationalPeriodTotal + q}{d})
\end{align*}
which then equals
\begin{equation}
 \sum_{\beta \in \set{0,1}^{|A_d|}} 
		\pm
    \exp\left(
      	\commonToAngle I
        \sum_{j \in A_d} -1 ^ {\beta_{(j)}}
          \numbToCommon \rationalAngle{p \rationalPeriodTotal + q}
      \right)
\end{equation}
where $\beta$ can be though of as a bit string deciding whether the angle from seed
$j \in A_d$ is added to or subtracted from the exponent.

Then, we have that
% TODO: update with common base stuff...
\begin{align*}
 \changeTotalOneDim =&
  \pm 2^{-|A|} \sum_{p= 0}^{\frac{\totalPeriod}{\rationalPeriodTotal} - 1} \sum_{q = 0}^{\rationalPeriodTotal - 1}
 \sum_{\beta \in \set{0,1}^{|A_d|}} 
 	 \pm
    \exp\left(
      	\commonToAngle I
        \sum_{j \in A_d} -1 ^ {\beta_{(j)}}
        \numbToCommon \rationalAngle{p \rationalPeriodTotal + q}
      \right)\\
 =&
  \pm 2^{-|A|} 
 \sum_{\beta \in \set{0,1}^{|A_d|}} 
		\pm
     \sum_{p= 0}^{\frac{\totalPeriod}{\rationalPeriodTotal} - 1} \sum_{q = 0}^{\rationalPeriodTotal - 1}
    \exp\left(
      	\commonToAngle I
        \sum_{j \in A_d} -1 ^ {\beta_{(j)}}
        \numbToCommon \rationalAngle{p \rationalPeriodTotal + q}
      \right).
\end{align*}

\newcommand{\eqWTSInnerProd}{
  \sum_{p= 0}^{\frac{\totalPeriod}{\rationalPeriodTotal} - 1} \sum_{q = 0}^{\rationalPeriodTotal - 1}
    \exp\left(
      	\commonToAngle I 
        \sum_{j \in A_d} -1 ^ {\beta_{(j)}}
        \numbToCommon 
        \rationalAngle{p \rationalPeriodTotal + q}I
      \right)}

Then, lets fix some $\beta \in \set{0, 1}^{|A_d|}$, define $Q$ such that
\begin{equation}
\label{eq:wts01}  
  Q = \eqWTSInnerProd.
\end{equation}
We will simplify $Q$ to show 2 distinct cases where $Q = 0$ for any choice of $\beta$.

Observe that 
\begin{align}
  \exp(\rationalAngle{p \rationalPeriodTotal + q}I)
  &= \exp\left(
      p \cdot \digSumPeriodic +
      \sum_{\ell = p \rationalPeriodTotal}^{p \rationalPeriodTotal + q}\mathrm{digit}(n^j, b^j, d^j, \ell)\right)\tag{by lemma \ref{lemma:angleBreakdown}} \\
      \label{eq:expBreakdown}
      &= \exp\left(p \cdot \digSumPeriodic\right) \exp\left(\sum_{\ell = 0}^{q}\mathrm{digit}(n, b, d, \ell)\right).
\end{align}

So then, by equation \eqref{eq:expBreakdown}, we get that
\begin{align}
&\exp\left(
    \sum_{j \in A_d} -1 ^ {\beta_{(j)}}
    \cdot
    \numbToCommon \cdot
    \rationalAngle{p \rationalPeriodTotal + q}I
\right)\notag \\ 
\label{eq:decomposeExp}
=&
\exp\left(
    I \sum_{j \in A_d} -1 ^ {\beta_{(j)}}
    \cdot
    \numbToCommon
    \cdot p \cdot
    \digSumPeriodic
\right)
\exp\left(
    I \sum_{j \in A_d} -1 ^ {\beta_{(j)}} \numbToCommon
      \sum_{\ell = 0}^{q}\mathrm{digit}(n^j, b^j, d^j, \ell)
\right).
\end{align}

% TODO: careful read through and w/ I.
% TODO: clean up and fix the lack of $2pi$ correction factor

We then use \eqref{eq:decomposeExp} to show that $Q$ equals
\begin{equation}
\label{eq:inner-outer-split}
  \sum_{p= 0}^{\frac{\totalPeriod}{\rationalPeriodTotal} - 1} \left[
    I \exp\left(
      pI \commonToAngle \sum_{j \in A_d} -1 ^ {\beta_{(j)}}
      \numbToCommon
      \digSumPeriodic
    \right)
  \left(
  I \sum_{q = 0}^{\rationalPeriodTotal - 1}
    \exp\left(
        \commonToAngle
        \sum_{j \in A_d} -1 ^ {\beta_{(j)}} \numbToCommon
          \sum_{\ell = 0}^{q}\mathrm{digit}(n^j, b^j, d^j, \ell)
    \right)\right)\right].
\end{equation}

\subsubsection*{Case 1: $\sum_{j \in A_d} (-1) ^ {\beta_{(j)}} \numbToCommon \digSumPeriodic \not\equiv 0 \mod \commonBase$}
Define
$$
  C_\beta = \sum_{q = 0}^{\rationalPeriodTotal - 1}
  \exp\left(
      \commonToAngle
      \sum_{j \in A_d} -1 ^ {\beta_{(j)}} \numbToCommon
        \sum_{\ell = 0}^{q}\mathrm{digit}(n, b, d, \ell)
  \right).
$$
Moreover, note that 
\begin{align*}
  &\exp\left(
      pI \commonToAngle \sum_{j \in A_d} -1 ^ {\beta_{(j)}}
      \numbToCommon
      \digSumPeriodic
    \right)
  = \prod_{j \in A_d} \exp\left(
      -1^{\beta_{(j)}}
     \cdot
    	\numbToAngle
     \cdot
      pI \cdot \digSumPeriodic 
    \right)
\end{align*}
and that  because $-1^{\beta_{(j)}} \cdot \numbToAngle pI \cdot \digSumPeriodic \equiv 0 \mod b^j$
\begin{equation*}
   -1^{\beta_{(j)}} \cdot \commonToAngle \numbToCommon pI \cdot \digSumPeriodic = \alpha \cdot 2\pi
\end{equation*}
for some $\alpha \in \N$. So,
\begin{equation*}
  \exp\left(-1^{\beta_{(j)}} \cdot\numbToAngle \cdot pI \cdot \digSumPeriodic \right) = \exp\left(0\right) = 1
\end{equation*}
when $p = \frac{\totalPeriod}{\rationalPeriodTotal}$. 

Because $\digSumPeriodic$ is a constant, we can conclude that
\begin{equation*}
 \exp\left(
      I \commonToAngle \sum_{j \in A_d} -1 ^ {\beta_{(j)}}
      \numbToAngle
      \digSumPeriodic
    \right) 
\end{equation*}
is a $\totalOverRationalFrac^{th}$ root of unity iff
$$\sum_{j \in A_d} -1 ^ {\beta_{(j)}}
      \numbToCommon
      \digSumPeriodic \not\equiv 0 \mod \commonBase
$$

So, for $\sum_{j \in A_d} -1 ^ {\beta_{(j)}}\numbToCommon \digSumPeriodic \neq 0$, we have that
\begin{align*}
  \eqWTSInnerProd &= C_\beta \sum_{p= 0}^{\frac{\totalPeriod}{\rationalPeriodTotal} - 1}
    \exp\left(
      pI \commonToAngle \sum_{j \in A_d} -1 ^ {\beta_{(j)}}
      \numbToCommon
      \digSumPeriodic
    \right) \\
    &= C_\beta \sum_{p = 0}^{\totalOverRationalFrac - 1}\exp\left(W_{\totalOverRationalFrac}^p\right) \\
    &= 0.
\end{align*}
where $W_{\totalOverRationalFrac}^p$ is the $\totalOverRationalFrac^{th}$ root of unity.

\subsubsection*{Case 2: $\sum_{j \in A_d} -1 ^ {\beta_{(j)}} \digSumPeriodic = 0$}
If $\sum_{j \in A_d} -1 ^ {\beta_{(j)}} \digSumPeriodic = 0$, then 
\begin{align*}
  \eqWTSInnerProd &= C_\beta \sum_{p= 0}^{\frac{\totalPeriod}{\rationalPeriodTotal} - 1} \exp(0) \\
  &= C_\beta.
\end{align*}
So $$\eqWTSInnerProd = 0$$ if $C_\beta = 0$.

\subsubsection*{To conclude}
If, $\forall \beta \in \set{0, 1}^{|A_d|}$, $\sum_{j \in A_d} -1 ^ {\beta_{(j)}} \neq 0$
or $C_\beta = 0$, then
\begin{align*}
  \changeTotalOneDim &= \sum_{i = 1}^T \combSingleTerm{d} \\&=  2^{-|A|}
  \sum_{\beta \in \set{0,1}^{|A_d|}} 
      \sum_{p= 0}^{\frac{\totalPeriod}{\rationalPeriodTotal} - 1} \sum_{q = 0}^{\rationalPeriodTotal - 1}
     \exp\left(
       \commonToAngle
         \sum_{j \in A_d} -1 ^ {\beta_{(j)}}
         \numbToCommon
         \rationalAngle{p \rationalPeriodTotal + q}I
       \right) \\
        &= 0.
\end{align*}
If the above is true for all $d \in D$, then 
$\Delta P_{0, T} = 0$.
