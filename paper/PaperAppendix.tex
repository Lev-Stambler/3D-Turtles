Before getting to the main result, we need to first prove the following lemma
\begin{lemma}{For all $j \in [k]$ and $x, y \in \N$ where $y < \rationalPeriodTotal$, we have that 
  $$
    \theta_{xT' + y}^j = x\cdot \digSumPeriodic + \sum_{q = 0}^y \digit(n, b, d, q)
  $$}
  \label{lemma:angleBreakdown}
  \begin{proof}
    We can then see that for $(n, b, d, \theta_{x\rationalPeriodTotal + y}^j) \in \pmb{s}_{x\rationalPeriodTotal + y}$,
    \begin{align*}
      \rationalAngle{x \rationalPeriodTotal + y} &= \sum_{i = 0}^{x \rationalPeriodTotal + y} \mathrm{digit}(n, b, d, i) \\
      &= \sum_{p = 0}^{(x- 1)\rationalPeriodTotal} \sum_{q=0}^{\rationalPeriodTotal - 1} \mathrm{digit}(n, b, d, p\rationalPeriodTotal + q)
          + \sum_{q = x \rationalPeriodTotal}^{x \rationalPeriodTotal + y} \mathrm{digit}(n, b, d, q)\\
      &= x \cdot \digSumPeriodic + \sum_{q = x \rationalPeriodTotal}^{\rationalPeriodTotal + y}\mathrm{digit}(n, b, d, q)\\
      &= x \cdot \digSumPeriodic + \sum_{q = 0}^{y}\mathrm{digit}(n, b, d, q)
    \end{align*}
    because $\mathrm{digit}(n, b, d, x\rationalPeriodTotal + \ell) = \mathrm{digit}(n, b, d, \ell)$ for any $\ell \in \N$
    by definition of periodicity.
  \end{proof}
\end{lemma}

\subsubsection{Closing along a single dimension}
Let $\changeTotalOneDim$ be the change of position along dimension $d$
from timestep 0 to $T$.
We are now ready to determine if we ``close" along one dimension. I.e.\ does
$
\changeTotalOneDim = 0
$?

Define $A_d = \set{j \mid j \in [k] \; \text{and}\; \inclIndic{d}{j} = 1}$, in other words,
$A_d$ is the set of rational parameters which are included in determining the position along the $d$th dimension.
We can then see that
\begin{align*}
\changeTotalOneDim &=\sum_{i = 1}^T \combSingleTerm{d}\\
  &=\sum_{i = 1}^T \prod_{j = 1}^k \left(\half \left(\exp\left({\numbToAngle \rationalAngle{i} I}\right) +  \exp\left(-\numbToAngle \rationalAngle{i} I\right)\right)\right)^\inclIndic{d}{j}\\
  &= 2^{-|A|} \sum_{p= 0}^{\frac{\totalPeriod}{\rationalPeriodTotal} - 1} \sum_{q = 0}^{\rationalPeriodTotal - 1}
    \prod_{j \in A_d} \left(
      \exp\left(\numbToAngle\rationalAngle{p \rationalPeriodTotal + q} I\right) +  \exp\left(-\numbToAngle\rationalAngle{p \rationalPeriodTotal + q} I\right)\right)
\end{align*}
by the Euler form of $\cos$.

\newcommand{\periodFrac}{\frac{\totalPeriod}{\rationalPeriod}}
\newcommand{\periodFracRational}{\frac{\rationalPeriodTotal}{\rationalPeriod}}

% TODO: not entirely correct with d
Next, observe that 
\begin{align*}
  &\prod_{j \in A_d} \left(
      \exp\left(\numbToAngle\rationalAngle{p \rationalPeriodTotal + q} I\right) +  \exp\left(-\numbToAngle\rationalAngle{p \rationalPeriodTotal + q} I\right)\right)\\
 =& 
  \exp(\numbToAngleNoJ{1}\rationalAngleJ{p \rationalPeriodTotal + q}{1} + \numbToAngleNoJ{2}\rationalAngleJ{p \rationalPeriodTotal + q}{2} + ... + \numbToAngleNoJ{d}\rationalAngleJ{p \rationalPeriodTotal + q}{d}) 
  + \exp(\numbToAngleNoJ{1}\rationalAngleJ{p \rationalPeriodTotal + q}{1} - \numbToAngleNoJ{2}\rationalAngleJ{p \rationalPeriodTotal + q}{2} + ... + \numbToAngleNoJ{d}\rationalAngleJ{p \rationalPeriodTotal + q}{d}) + ... \\
  &+ \exp(-\numbToAngleNoJ{1}\rationalAngleJ{p \rationalPeriodTotal + q}{1} - \numbToAngleNoJ{2}\rationalAngleJ{p \rationalPeriodTotal + q}{2} - ... - \numbToAngleNoJ{d}\rationalAngleJ{p \rationalPeriodTotal + q}{d})
\end{align*}
which then equals
\begin{equation}
 \sum_{\beta \in \set{0,1}^{|A_d|}} 
    \exp\left(
      	\commonToAngle I
        \sum_{j \in A_d} -1 ^ {\beta_{(j)}}
          \numbToCommon \rationalAngle{p \rationalPeriodTotal + q}
      \right)
\end{equation}
where $\beta$ can be though of as a bit string deciding whether the angle from seed
$j \in A_d$ is added to or subtracted from the exponent.

Then, we have that
% TODO: update with common base stuff...
\begin{align*}
 \changeTotalOneDim =&
  2^{-|A|} \sum_{p= 0}^{\frac{\totalPeriod}{\rationalPeriodTotal} - 1} \sum_{q = 0}^{\rationalPeriodTotal - 1}
 \sum_{\beta \in \set{0,1}^{|A_d|}} 
    \exp\left(
        \sum_{j \in A_d} -1 ^ {\beta_{(j)}}
        \rationalAngle{p \rationalPeriodTotal + q}I
      \right)\\
 =&
  2^{-|A|} 
 \sum_{\beta \in \set{0,1}^{|A_d|}} 
     \sum_{p= 0}^{\frac{\totalPeriod}{\rationalPeriodTotal} - 1} \sum_{q = 0}^{\rationalPeriodTotal - 1}
    \exp\left(
        \sum_{j \in A_d} -1 ^ {\beta_{(j)}}
        \rationalAngle{p \rationalPeriodTotal + q}I
      \right).
\end{align*}

\newcommand{\eqWTSInnerProd}{
  \sum_{p= 0}^{\frac{\totalPeriod}{\rationalPeriodTotal} - 1} \sum_{q = 0}^{\rationalPeriodTotal - 1}
    \exp\left(
        \sum_{j \in A_d} -1 ^ {\beta_{(j)}}
        \rationalAngle{p \rationalPeriodTotal + q}I
      \right)}

Then, lets fix some $\beta \in \set{0, 1}^{|A_d|}$, define $Q$ such that
\begin{equation}
\label{eq:wts01}  
  Q = \eqWTSInnerProd.
\end{equation}
We will simplify $Q$ to show 2 distinct cases where $Q = 0$ for any choice of $\beta$.

Observe that 
\begin{align}
  \exp(\rationalAngle{p \rationalPeriodTotal + q}I)
  &= \exp\left(p \cdot \digSumPeriodic +
    \numbToCommon \sum_{\ell = p \rationalPeriodTotal}^{p \rationalPeriodTotal + q}\mathrm{digit}(n^j, b^j, d^j, \ell)\right) \tag{by lemma \ref{lemma:angleBreakdown}} \\
    \label{eq:expBreakdown}
    &= \exp\left(p \cdot \digSumPeriodic\right) \exp\left(\numbToCommon \sum_{\ell = 0}^{q}\mathrm{digit}(n, b, d, \ell)\right).
\end{align}

So then, by equation \eqref{eq:expBreakdown}, we get that
\begin{align}
&\exp\left(
    \sum_{j \in A_d} -1 ^ {\beta_{(j)}}
    \rationalAngle{p \rationalPeriodTotal + q}I
\right)\notag \\ 
\label{eq:decomposeExp}
=&
\exp\left(
    \sum_{j \in A_d} -1 ^ {\beta_{(j)}}
    \cdot p \cdot
    \digSumPeriodic
\right)
\exp\left(
    \sum_{j \in A_d} -1 ^ {\beta_{(j)}} \numbToCommon
      \sum_{\ell = 0}^{q}\mathrm{digit}(n^j, b^j, d^j, \ell).
\right).
\end{align}

We then use \eqref{eq:decomposeExp} to show that $Q$ equals
\begin{equation}
\label{eq:inner-outer-split}
  \sum_{p= 0}^{\frac{\totalPeriod}{\rationalPeriodTotal} - 1} \left[
    \exp\left(
      pI \sum_{j \in A_d} -1 ^ {\beta_{(j)}}
      \digSumPeriodic
    \right)
  \left(
  \sum_{q = 0}^{\rationalPeriodTotal - 1}
    \exp\left(
        \sum_{j \in A_d} -1 ^ {\beta_{(j)}} \numbToCommon
          \sum_{\ell = 0}^{q}\mathrm{digit}(n^j, b^j, d^j, \ell)
    \right)\right)\right].
\end{equation}

\subsubsection*{Case 1}
Define
$$
  C_\beta = \sum_{q = 0}^{\rationalPeriodTotal - 1}
  \exp\left(
      \sum_{j \in A_d} -1 ^ {\beta_{(j)}} \numbToCommon
        \sum_{\ell = 0}^{q}\mathrm{digit}(n, b, d, \ell)
  \right).
$$
Moreover, note that 
\begin{align*}
  &\exp\left(
      pI \sum_{j \in A_d} -1 ^ {\beta_{(j)}}
      \digSumPeriodic
    \right)
  = \prod_{j \in A_d} \exp\left(-1^{\beta_{(j)}} pI \cdot \digSumPeriodic \right)
\end{align*}
and that 
\begin{equation*}
  \exp\left(-1^{\beta_{(j)}} pI \cdot \digSumPeriodic \right) = \exp\left(0\right) = 1
\end{equation*}
when $p = \frac{\totalPeriod}{\rationalPeriodTotal}$. So, we can see that
$$
\prod_{j \in A_d} \exp\left(-1^{\beta_{(j)}} pI \cdot \digSumPeriodic \right) = 1
$$
when $p = \frac{\totalPeriod}{\rationalPeriodTotal}$.

Because $\digSumPeriodic$ is a constant, we can conclude that
\begin{equation*}
 \exp\left(
      I \sum_{j \in A_d} -1 ^ {\beta_{(j)}}
      \digSumPeriodic
    \right) 
\end{equation*}
is a $\totalOverRationalFrac^{th}$ root of unity iff 
$$\sum_{j \in A_d} -1 ^ {\beta_{(j)}}\digSumPeriodic \neq 0$$

So, for $\sum_{j \in A_d} -1 ^ {\beta_{(j)}}\digSumPeriodic \neq 0$, we have that
\begin{align*}
  \eqWTSInnerProd &= C_\beta \sum_{p= 0}^{\frac{\totalPeriod}{\rationalPeriodTotal} - 1}
    \exp\left(
      pI \sum_{j \in A_d} -1 ^ {\beta_{(j)}}
      \digSumPeriodic
    \right) \\
    &= C_\beta \sum_{p = 0}^{\totalOverRationalFrac - 1}\exp\left(W_{\totalOverRationalFrac}^p\right) \\
    &= 0.
\end{align*}
where $W_{\totalOverRationalFrac}^p$ is the $\totalOverRationalFrac^{th}$ root of unity.

If $\sum_{j \in A_d} -1 ^ {\beta_{(j)}} \digSumPeriodic = 0$, then 
\begin{align*}
  \eqWTSInnerProd &= C_\beta \sum_{p= 0}^{\frac{\totalPeriod}{\rationalPeriodTotal} - 1} \exp(0) \\
  &= C_\beta.
\end{align*}

So, we have now shown that \eqref{eq:wts01} is always $0$ or $1$. In particular, we also have our first 
closure result. If, $\forall \beta \in \set{0, 1}^{|A_d|}$, $\sum_{j \in A_d} -1 ^ {\beta_{(j)}} \neq 0$,
\begin{align*}
  \sum_{i = 1}^T \combSingleTerm{d} &=  2^{-|A|} 
  \sum_{\beta \in \set{0,1}^{|A_d|}} 
      \sum_{p= 0}^{\frac{\totalPeriod}{\rationalPeriodTotal} - 1} \sum_{q = 0}^{\rationalPeriodTotal - 1}
     \exp\left(
         \sum_{j \in A_d} -1 ^ {\beta_{(j)}}
         \rationalAngle{p \rationalPeriodTotal + q}I
       \right) \\
        &= 0
\end{align*}
for all $d \in D$.
So then, $\Delta P_{0, T} = 0$ whenever 
no degree $1$ multinomials with coefficients of $-1$ or $1$ and $|A_d|$ variables has a root of $(\digSumPeriodicNoJ^1, \digSumPeriodicNoJ^2, ..., \digSumPeriodicNoJ^{|A_d|})$.
In other words, if variables, $x_1 = \digSumPeriodicNoJ^1, x_2 = \digSumPeriodicNoJ^2, ...$, there does not exist a multinomial
of form 
$$
  \pm x_1 \pm x_2 \pm ... \pm x_{|A_d|} = 0.
$$

Said differently, let multinomial $m: \Z_{\commonBase} \rightarrow \Z_{\commonBase}$ be such that
\begin{equation*}
  m(x_1, x_2, ..., x_{|A_d|}) = \prod_{\beta \in \set{0, 1}^{|A_d|}} \left(\sum_{j \in A_d} -1^{\beta_j}AAA \right).
\end{equation*}
Then, if $m(\digSumPeriodicNoJ^1, \digSumPeriodicNoJ^2, ..., \digSumPeriodicNoJ^{|A_d|}) \neq 0$,
we know that the shape must close.


\subsubsection*{There does exist}

% TODO: replace this eqWTSInnerProf with like P_{}(d) or smthng...
In the case that $\exists \beta \in \set{0, 1}^{|A_d|}$ such that $\sum_{j \in A_d} -1 ^ {\beta_{(j)}} = 0$, it is still possible
for
$$
\eqWTSInnerProd = 0
$$
if $C_\beta = 0$. So, 
\begin{equation*}
  \sum_{q = 0}^{\rationalPeriodTotal - 1}
  \exp\left(
      \sum_{j \in A_d} -1 ^ {\beta_{(j)}}
        \sum_{\ell = 0}^{q}\mathrm{digit}(n, b, d, \ell)
  \right) = 0
\end{equation*}

